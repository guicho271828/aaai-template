% In this section, we discuss various approximations of delete-relaxed optimal cost $h^+(s)$.

A domain-independent heuristic function $h$ in classical planning is
a function of a state $s$ and the problem $\brackets{P,A,I,G}$,
but the notation $h(s)$ usually omits the latter.
In addition to what we discussed in the main article, this section also uses a notation $h(s,G)$.
It returns an estimate of the cumulative cost from $s$ to one of the goal states (states that satisfy $G$),
typically through a symbolic, non-statistical means including problem relaxation and abstraction.
Notable \lsota functions that appear in this paper includes
$\ff, \hmax, \ad, \gc$ \cite{hoffmann01,bonet2001planning,FikesHN72}.

A significant class of heuristics is called delete relaxation heuristics,
which solve a relaxed problem which does not contain delete effects,
and then returns the cost of the solution of the relaxed problem as an output.
The cost of the optimal solution of a delete relaxed planning problem from a state $s$ is
denoted by $h^+(s)$, but this is too expensive to compute in practice (NP-complete) \cite{bylander1996}.
Therefore, practical heuristics typically try to obtain its further relaxations
that can be computed in polynomial time.

One such admissible heuristic based on delete-relaxation is called
$\hmax$ \cite{bonet2001planning} that is recursively defined as follows:

\begin{align}
 \hmax(s,G) = \max_{p\in G}
 \left\{
  \begin{array}{l}
   0\ \text{if}\ p\in s.\ \text{Otherwise,}\\
   \min_{\braces{a\in A\mid p\in\adde(a)}} \\
    \quad \left[\cost(a)+\ad(s, \pre(a))\right].
  \end{array}
 \right.
\end{align}

Its inadmissible variant is called
additive heuristics $\ad$ \cite{bonet2001planning} that is recursively defined as follows:

\begin{align}
 \ad(s,G) = \sum_{p\in G}
 \left\{
  \begin{array}{l}
   0\ \text{if}\ p\in s.\ \text{Otherwise,}\\
   \min_{\braces{a\in A\mid p\in\adde(a)}} \\
    \quad \left[\cost(a)+\ad(s, \pre(a))\right].
  \end{array}
 \right.
\end{align}

Another inadmissible delete-relaxation heuristics called
$\ff$ \cite{hoffmann01} is defined based on another heuristics $h$, such as $h=\ad$, as a subprocedure.
For each unachieved subgoal $p\in G\setminus s$,
% the action $a=\argmin_{\braces{a\in A\mid p\in\adde(a)}} \left[\cost(a)+h(s, \pre(a))\right]$
the action $a$ that adds $p$ with the minimal $\left[\cost(a)+h(s, \pre(a))\right]$
% The action $a$ which minimizes the second case ($p\not\in S$) of the definition above
is conceptually ``the cheapest action that achieves a subgoal $p$ for the first time under delete relaxation'',
called the \emph{cheapest achiever} / \emph{best supporter} $\text{bs}(p,s,h)$ of $p$.
$\ff$ is defined as the sum of actions in a relaxed plan $\Pi^+$ constructed as follows:

\begin{align}
 \ff(s,G,h) &= \sum_{a\in \Pi^+(s,G,h)} \cost(a)\\
 \Pi^+(s,G,h) &= \bigcup_{p\in G}
 \left\{
  \begin{array}{l}
   \emptyset\ \text{if}\ p\in s.\ \text{Otherwise,}\\
   \braces{a} \cup \Pi^+(s,\pre(a)) \\
   \qquad \text{where}\ a=\text{bs}(p,s,h).
  \end{array}
 \right.\\
 \text{bs}(p,s,h)&=\argmin_{\braces{a\in A\mid p\in \adde(a)}} \left[\cost(a)+h(s, \pre(a))\right].
\end{align}

Goal Count heuristics $\gc$ is a simple heuristic proposed in \cite{FikesHN72}
that counts the number of propositions that are not satisfied yet.
$\brackets{\text{condition}}$ is a cronecker's delta / indicator function that returns 1 when the condition is satisfied.
\begin{align}
 \gc(s,G) &= \sum_{p\in G} \dbrackets{p\not \in s}.
\end{align}

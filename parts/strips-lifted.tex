Planning Domain Description Language (PDDL) is a modeling language for
a Lifted STRIPS planning formalism \citep{FikesHN72} and its extensions \citep{pddlbook}.
Let $\mathcal{F}(T)$ be a formula consisting of
 logical operations $\braces{\land,\lnot}$
 % constants $\braces{\top,\bot}$, and
and a set of terms $T$.
For example, when $T=\{\textit{have}(\textit{I}, \textit{food}), \textit{full}(\textit{I})\}$,
 then $\textit{have}(\textit{I}, \textit{food})\land\lnot\textit{full}(\textit{I})\in \mathcal{F}(T)$.
 % 
We denote a lifted STRIPS planning problem
as a 5-tuple $\brackets{O,P,A,I,G}$.
 $O$ is a set of objects ($\ni\textit{food}$),
 $P$ is a set of predicates ($\ni\textit{full}(x)$), and
 $A$ is a set of lifted actions ($\ni\textit{eat}$).
Each predicate $p\in P$ has an arity $\ar{p}\geq 0$.
Predicates are instantiated/\emph{grounded} into
propositions $P(O)=\bigcup_{p\in P} \parens{\braces{p}\times O\times\overset{\ar{p}}{\ldots}\times O}$,
% where each proposition is denoted as $p(o_1,\cdots,o_{\ar{p}})$, 
such as \textit{have(I, food)}.
A state $s\subseteq P(O)$ represents truth assignments to the propositions,
e.g., $s=\braces{\textit{have(I, food)}}$ represents \textit{have(I, food)} = $\top$. % and \textit{full(I)} = $\bot$.
We can also represent it as a bitvector of size $\sum_p O^{\ar{p}}$.

Each lifted action $a(X)\in A$ has an arity $\ar{a}$
and parameters $X=(x_1,\cdots,x_{\ar{a}})$, such as \textit{eat($x_1, x_2$)}.
% Without loss of generality, we assume $\adde{a} \cap \dele{a} = \emptyset$ by removing the duplicates from $\dele{a}$.
Lifted actions are instantiated into 
\emph{ground actions} $A(O)=\bigcup_{a\in A} \parens{\braces{a}\times O\times\overset{\ar{a}}{\ldots}\times O}$,
such as $\textit{eat}(\textit{I},\textit{food})$.
%, substituting $x_1,x_2$ with \textit{I}, \textit{food}.
% where $a(o_1,\cdots,o_{\ar{a}})$ substitutes the parameters $(x_1,\cdots,x_{\ar{a}})$ in $a(X)$ with the objects $(o_1,\cdots,o_{\ar{a}})$.
$a(X)$ is a 3-tuple $\brackets{\pre{a},\adde{a},\dele{a}}$, where
 $\pre{a}, \adde{a}, \dele{a} \in \mathcal{F}(P(X))$ are
 preconditions, add-effects, and delete-effects:
e.g., \textit{eat($x_1$, $x_2$)} = $\brackets{\braces{\textit{have}(x_1, x_2)}, \braces{\textit{full}(x_1)}, \braces{\textit{have}(x_1,x_2)}}$.
The semantics of these three elements are as follows:
A ground action $a^{\dagger}\in A(O)$ is \emph{applicable} when a state $s$ \emph{satisfies} $\pre{a^{\dagger}}$, i.e., $\pre{a^{\dagger}}\subseteq s$,
and applying an action $a^{\dagger}$ to $s$ yields a new successor state
$a^{\dagger}(s) = (s \setminus \dele{a^{\dagger}}) \cup \adde{a^{\dagger}}$,
e.g., $\textit{eat}(\textit{I},\textit{food})=$ ``\emph{I can eat a food when I have one, and if I eat one I am full but the food is gone.}''
% By construction, the learned representation ensures $\adde{a} \cap \dele{a} = \emptyset$. % should discussed in the cube-space AE section
% The \emph{STRIPS subclass} of the Classical Planning problems
% is those problems where all effects are conditioned by $\top$.
Finally,
$I,G\subseteq P(O)$ are the initial state and a goal condition, respectively.
The task of classical planning is to find a \emph{plan} $(a^{\dagger}_1,\cdots,a^{\dagger}_n)$
which satisfies $a^{\dagger}_n \circ \cdots \circ a^{\dagger}_1(I) \subseteq G$.

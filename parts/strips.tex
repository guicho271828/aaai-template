Let $\mathcal{F}(V)$ be a propositional formula consisting of
 logical operations $\braces{\land,\lnot}$,
 constants $\braces{\top,\bot}$, and
 a set of propositional variables $V$.
We define a grounded (propositional) STRIPS Planning problem
as a 4-tuple $\brackets{P,A,I,G}$
where
 $P$ is a set of propositions,
 $A$ is a set of actions,
 $I\subseteq P$ is the initial state, and
 $G\subseteq P$ is a goal condition.
Each action $a\in A$ is a 3-tuple $\brackets{\pre{a},\adde{a},\dele{a}}$ where
 $\pre{a} \in \mathcal{F}(P)$ is a precondition and
 $\adde{a}$, $\dele{a}$ are the add-effects and delete-effects, where $\adde{a} \cap \dele{a} = \emptyset$.
%Each effect is denoted as $c \triangleright e$ where
% $c \in \mathcal{F}(P)$ is an \emph{effect condition} and
% $e \in P$.
A state $s\subseteq P$ is a set of true propositions,
an action $a$ is \emph{applicable} when $s$ \emph{satisfies} $\pre{a}$,
and applying an action $a$ to $s$ yields a new successor state
$a(s) = (s \setminus \dele{a}) \cup \adde{a}$.
% By construction, the learned representation ensures $\adde{a} \cap \dele{a} = \emptyset$. % should discussed in the cube-space AE section
% The \emph{STRIPS subclass} of the Classical Planning problems
% is those problems where all effects are conditioned by $\top$.

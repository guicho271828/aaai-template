SAS+ formalism \cite{backstrom1995complexity} is an alternative formalism to propositional STRIPS planning.
A SAS+ planning problem is a 4-tuple $\brackets{V,A,I,G}$
where
$V$ is a set of SAS+ state variables,
$A$ is a set of ground actions,
$I$ is the initial state,
and
$G$ is the goal condition.
% 
Each state variable $v$ has an integer value $i\in [0, M_v-1]=d(v)$,
where $d(v)$ denotes the domain of $v$.
% 
Each action $a\in A$ is a tuple $\brackets{\pre(a), \eff(a)}$,
where in SAS+
both $\pre(a)$ and $\eff(a)$ are sets of key-value pairs
in the form of $\{\ldots, (v,i), \ldots\}$ with keys $v\in V$ and values $i\in d(v)$.
% 
A state $s$ is a vector that specifies the values of all variables in $V$.
$\pre(a)$ and $\eff(a)$ may not specify some variables, thus are called \emph{partial states}.
An action $a$ is applicable in a state $s$ when its variable assignments match those in $\pre(a)$.
% i.e., $\forall (v,i)\in\pre(a); \forall (v,j)\in s; i=j$.
An action produces a successor state $s'=a(s)$
by overwriting some variables in $s$ using the values specified in $\eff(a)$.
For example,
% if $s=\{(0,1),(1,1),(2,2)\}$, $\pre(a)=\{(1,1)\}$ and $\eff(a)=\{(0,0),(1,0)\}$, then
% $a$ is applicable, and $a(s)=\{(0,0),(1,0),(2,2)\}$.
if $s=[1,1,2]$, $\pre(a)=\{(1,1)\}$ and $\eff(a)=\{(0,0),(1,0)\}$, then
$a$ is applicable because $s[1]=1$, and $a(s)=[0,0,2]$ because it updates $s$ as $s[0]\gets 0$ and $s[1]\gets 0$.
% 
% Extensions on axioms and conditional effects are straightforward.
% 
The goal condition $G$ is a partial state.

Each integer $i$ of a variable $v$ usually corresponds to a propositional variable $p\in F$ in STRIPS.
The propositional variables represented by a single SAS+ variable $v$
are mutex to each other, i.e., no two propositional variables become true at the same time in any state.
When $i$ does not correspond to a proposition,
it is usually $i=0$ that represents \emph{none-of-the-above}, i.e.,
for a set of propositions $p_1, p_2, \ldots p_{M_v-1}$ represented by each $i\in [1, M_v-1]$,
$i=0$ represents $\lnot p_1\land \ldots \land \lnot p_{M_v-1}$.
Such mutex contraints are discovered by the PDDL-to-SAS+ translator phase of Fast Downward \cite{}.
% 
Since STRIPS and SAS+ are equally expressive formalism, mapping SAS+ back to STRIPS is straightforward:
Each value $i$ of a SAS+ variable $v$ maps to an index $i+\sum_{j=0}^{v-1}M_j$.

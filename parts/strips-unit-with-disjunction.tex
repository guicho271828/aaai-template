Let $\mathcal{F}(V)$ be a propositional formula consisting of
 logical operations $\braces{\land,\lor,\lnot}$,
 constants $\braces{\top,\bot}$, and
 a set of propositional variables $V$.
We define a grounded (propositional) Classical Planning problem
as a 4-tuple $\brackets{P,A,I,G}$
where
 $P$ is a set of propositions,
 $A$ is a set of actions,
 $I\subset P$ is the initial state, and
 $G\subset P$ is a goal condition.
Each action $a\in A$ is a 3-tuple $\brackets{\pre(a),\adde(a),\dele(a)}$ where
 $\pre(a) \in \mathcal{F}(P)$ is a precondition and
 $\adde(a)$, $\dele(a)$ are the sets of effects called add-effects and delete-effects, respectively.
Each effect is denoted as $c \triangleright e$ where
 $c \in \mathcal{F}(P)$ is an \emph{effect condition} and
 $e \in P$.
A state $s\subseteq P$ is a set of true propositions,
an action $a$ is \emph{applicable} when $s \vDash \pre(a)$ ($s$ \emph{satisfies} $\pre(a)$),
and applying an action $a$ to $s$ yields a new successor state $a(s)$ which is
$a(s) = s \cup \braces{e \mid (c \triangleright e) \in \adde(a), c\vDash s} \setminus \braces{e \mid (c \triangleright e) \in \dele(a), c\vDash s}$.

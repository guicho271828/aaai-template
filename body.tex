% 
% 
% * Maintain 100 characters / line.
% * too much ``''s make the sentence look scattered and visually less recognizable. ``e.g.'' also.
% * \em, \bf, \it are all obsolete \TeX primitives, and it does not take effect properly ---
%   for example, {\bf {\it aaa}} shows ``aaa'' in italic but NOT IN BOLD.
%   Use \emph{}, \textit{}, \textbf{} and so on.
% 
% * Always use \ff, \fd, \cea, or other abbreviation commands in abbrev.sty.
%   Inconsistent notations would not be the major reason for rejection, but 
%   they eventually severely harm the impression of the paper.
% 
% * use of footnotes should be minimized.
% * IPC2011 should always be \ipc . The definition can later be modified in abbrev.sty .
% * prefer separated words over hyphened words. domain
%   independent>domain-independent, planner independent >
%   planner-independent.
% * Table, Figure, Fig., should not be used directly. Always use \refig and \reftbl. When the development flag is enabled, direct use of \ref signals an error.
% * Caption ends with a period.
% 
% 
\begin{abstract}
AAA!
\end{abstract}

% How to write an intuitive paper:

% 1. There is a villain.
% 2. The villain has a weakness.
% 3. The hero defeats this villain by attacking the weakness.

% A good weakness tends to be theoretical and fundamental.
% Practical limitations can be a weakness, but does not provide motivation as strong as theoretical one.


\section{\texttt{aaai-template}}\label{aaai-template}

For the frequent attendants of top-tier AI conferences!
This repository includes templates and makefiles for:

\begin{itemize}

\item
  AAAI style
\item
  ECAI style
\item
  IJCAI style
\item
  JAIR style
\item
  NIPS style
\item
  ICML style
\item
  ICLR style
\item
  CVPR style
\end{itemize}

Requirements: GNU Make, TexLive, inkscape, perl (see dependency.sh)

\textbf{Update:} AAAI Press recently made a significant change to the
camera-ready requirements (such as
https://www.aaai.org/Publications/Author/icaps-submit.php). To address
the requirement we changed the file structure --- see below.

\section{File structure}

\begin{itemize}

\item
  \texttt{main.tex}

  \begin{itemize}
  
  \item
    toplevel file for the main paper, only containing the preamble.
  \end{itemize}
\item
  \texttt{body.tex}

  \begin{itemize}
  
  \item
    the text inside
    \texttt{\textbackslash{}begin\{document\}}--\texttt{\textbackslash{}end\{document\}}
    for \texttt{main.tex}.
  \end{itemize}
\item
  \texttt{supplemental.tex}

  \begin{itemize}
  
  \item
    toplevel text file for the supplementary material. Unlike the main
    paper, the text should be in this file too.
  \item
    \texttt{supplemental.tex} and \texttt{main.tex} can cross-reference
    the figures, tables and sections each other.
  \end{itemize}
\item
  \texttt{common-header.sty}

  \begin{itemize}
  
  \item
    Part of the preamble shared by \texttt{main.tex} and
    \texttt{supplemental.tex}.
  \item
    This file also contains the specific code for each conference. You
    should uncomment the code for the conference you plan to submit
  \end{itemize}
\item
  \texttt{commands-general.sty}

  \begin{itemize}
  
  \item
    general custom commands
  \end{itemize}
\item
  \texttt{commands-abbrev.sty}

  \begin{itemize}
  
  \item
    custom commands (only for the abbreviation)
  \end{itemize}
\end{itemize}

\section{Usage notes}
\label{sec:usage}
\begin{itemize}
\item
  It encourages the use of Inkskape to
  prepare svg images in \texttt{img} subdirectory, which are
  automatically compiled into pdf figures. (This is a recommended way,
  since pngs are bitmaps and doesnt print nicely.)

  On OSX, inkscape is available from \texttt{brew\ install\ inkscape}
\item
  \texttt{make\ submission}, \texttt{make\ archive},
  \texttt{make\ arxiv} : These \texttt{make} targets will create a
  \texttt{submission} directory and prepares the camera-ready tex files.
  There are sometimes extensive instruction for preparing the
  camera-ready submission, such as
  https://www.aaai.org/Publications/Author/icaps-submit.php .

  \begin{itemize}
  
  \item
    These camera-ready submissions do not allow the use of
    \texttt{\textbackslash{}input\{\}} command. When you run
    \texttt{make\ submission}, the results generated in the directory
    will have

    \begin{itemize}
    
    \item
      a single, flattened tex file whose \texttt{\textbackslash{}input}
      commands are inlined completely
    \item
      All image files referenced by the text are renamed and put in this
      root directory (AAAI Press does not allow putting images in the
      nested subdirectories)
    \item
      Garbage files (log files etc.) are removed.
    \end{itemize}
  \item
    \textbf{Usage note}: all \texttt{\textbackslash{}input\{\}} commands
    must be at the beginning of line, nothing before or after it.
    Otherwise it may remove some necessary text
  \item
    \texttt{make\ archive} compresses the \texttt{submission/} directory
    and create a zip or a tar.gz file. AAAI Press receive the zip file
    only, but this feature is also useful when submitting to Arxiv.

    \begin{itemize}
    
    \item
      Style files are removed (they are not allowed).
    \end{itemize}
  \item
    \texttt{make\ arxiv} is same as \texttt{make\ archive}, but does not
    remove the style files.
  \end{itemize}
\item
  \texttt{make\ auto} watches the source files and builds the pdf when
  they are updated. Requirements: \texttt{inotify-tools} package (it
  uses \texttt{inotifywait} for watching files and also sends messages
  via inotify notification popup window)
\item
  This template also supports JSAI, a local Japanese non-refereed
  confernece.
 \item Do not use \texttt{\textbackslash{}ref} command directly like this: \ref{sec:usage}.
       This cause style inconsistency, e.g., Section \ref{sec:usage} vs. Sec. \ref{sec:usage}.
       Use \texttt{\textbackslash{}refsec}, \texttt{\textbackslash{}reftbl}, etc.
       in \texttt{common-general.sty}.
\end{itemize}



\section{If you have space in your paper, credit me}

Like this! \cite{aaai-template}
Or you can also use \texttt{nocite} command like this. \nocite{aaai-template}
Depending on the conference styles,
\texttt{natbib} may or may not be automatically loaded,
and may or may not be prohibited (due to incompatibility).
We thus define our version of \texttt{citet} and \texttt{citep},
only when they are not defined.
Like \citet{aaai-template}, or like this \citep{aaai-template}.



\section{Example Introduction}

[The text is borrowed from a famous latex template for Kakenhi Grant
Proposal (Japanese version of NSF in US) translated into English.]

The true purpose of this study is, in a nutshell, to fulfill the childhood dream of finding an elephant egg.

BBB! \refig{fig:ip} \cite{Asai2016}

\begin{figure}[tb]
 \includegraphics[width=\linewidth]{img/static/ip.png}
 \caption{This is Invasion Percolation}
 \label{fig:ip}
\end{figure}

\lmcut, \mands, \pdb, \ff, \ce, \cg, \ad, \lc heuristics.

In math mode,

\[
 \lmcut, \mands, \pdb, \ff, \ce, \cg, \ad, \lc.
\]

\lmcuto, \mandso, \ffo, \ceo, \cgo, \ado, \gco, \lco heuristics.

In math mode,

\[
 \lmcuto, \mandso, \ffo, \ceo, \cgo, \ado, \gco, \lco.
\]

% \begin{minted}{common-lisp}
% (defun factorial (n)
%   (if (zerop n)
%       1
%       (* n (factorial (1- n)))))
% \end{minted}

\section{Background}

The purpose of this research is
a multifaceted investigation of elephant egg shells from biological, chemical, theoretical, and engineering aspects.
Elephant egg shells weigh more than 80 kg.
It is necessary not only to support the weight of the elephant and its
nutrient source, the large mass of the yolk, but also the weight of the
parent elephant that warms the egg. For this reason, the elephant egg
shell is considered to have a structure that is entirely different
from that of bird's (Fig. 1) egg shell. Also, if the mechanism
of the elephant egg shell is clarified,

\begin{itemize}
 \item Elucidation of elephant ecology, understanding of dinosaur egg structure (biology),
 \item Elucidation of shell chemical formation reaction (chemistry),
 \item Research on the relationship between the atomic level structure of the shell and C60 and nanoclusters Research (physics),
 \item Artificially create an elephant shell and apply it to the body of a car (engineering)
\end{itemize}

The impact on science and society is immeasurable.

\section{Method}

We traveled around the world to find elephant eggs. This has also been a dream since I was a kid. 

\section{Empirical Evaluation}

Elephant eggs are phantastic. 
Elephant eggs are phantastic. 
Elephant eggs are phantastic. 
Elephant eggs are phantastic. 
Elephant eggs are phantastic. 
Elephant eggs are phantastic. 
Elephant eggs are phantastic. 
Elephant eggs are phantastic. 

\section{Related Work}

According to \citet{folbre1997future},
In ``Horton Hatches the Egg'' \cite{seuss1968horton},
a tired mother bird asked Horton the elephant to give her a break and
sit on the nest instead, which resulted in a chick with a small trunk and
elephant ears.
\citet{kipling1983elephant} discuss how the elephant got its long trunk.
\citet{cooper2001complete} analyzed the mitochondria DNA sequence of the extinct Elephant-Bird species,
such as \emph{Aepyornis maximus}.
\citet{carlqvist2003theory} studied the astronomical object called Twisted Elephant Trunks
based on the magnetic flux ropes between molecular clouds.

Existing work studied the eggs of the blue whale.
While blue whales and elephants are mammals, they live in completely different habitats.

\section{Conclusion}

We could not find Elephant's eggs.

\section*{Checklist}

Elephant egg ES cells are not cultured or elephant clones are not
generated. Since elephant individuals are not taken out of the field,
they do not conflict with the Washington Convention and the Convention
on Biological Diversity. In addition, since no recombination experiment
is conducted, it does not conflict with the Cartagena Protocol.

%%% BEGIN INSTRUCTIONS %%%
% The checklist follows the references.  Please
% read the checklist guidelines carefully for information on how to answer these
% questions.  For each question, change the default \answerTODO{} to \answerYes{},
% \answerNo{}, or \answerNA{}.  You are strongly encouraged to include a {\bf
% justification to your answer}, either by referencing the appropriate section of
% your paper or providing a brief inline description.  For example:
% \begin{itemize}
%   \item Did you include the license to the code and datasets? \answerYes{See Section~\ref{gen_inst}.}
%   \item Did you include the license to the code and datasets? \answerNo{The code and the data are proprietary.}
%   \item Did you include the license to the code and datasets? \answerNA{}
% \end{itemize}
% Please do not modify the questions and only use the provided macros for your
% answers.  Note that the Checklist section does not count towards the page
% limit.  In your paper, please delete this instructions block and only keep the
% Checklist section heading above along with the questions/answers below.
%%% END INSTRUCTIONS %%%

% \begin{enumerate}
% 
% \item For all authors...
% \begin{enumerate}
%   \item Do the main claims made in the abstract and introduction accurately reflect the paper's contributions and scope?
%     \answerTODO{}
%   \item Did you describe the limitations of your work?
%     \answerTODO{}
%   \item Did you discuss any potential negative societal impacts of your work?
%     \answerTODO{}
%   \item Have you read the ethics review guidelines and ensured that your paper conforms to them?
%     \answerTODO{}
% \end{enumerate}
% 
% \item If you are including theoretical results...
% \begin{enumerate}
%   \item Did you state the full set of assumptions of all theoretical results?
%     \answerTODO{}
% 	\item Did you include complete proofs of all theoretical results?
%     \answerTODO{}
% \end{enumerate}
% 
% \item If you ran experiments...
% \begin{enumerate}
%   \item Did you include the code, data, and instructions needed to reproduce the main experimental results (either in the supplemental material or as a URL)?
%     \answerTODO{}
%   \item Did you specify all the training details (e.g., data splits, hyperparameters, how they were chosen)?
%     \answerTODO{}
% 	\item Did you report error bars (e.g., with respect to the random seed after running experiments multiple times)?
%     \answerTODO{}
% 	\item Did you include the total amount of compute and the type of resources used (e.g., type of GPUs, internal cluster, or cloud provider)?
%     \answerTODO{}
% \end{enumerate}
% 
% \item If you are using existing assets (e.g., code, data, models) or curating/releasing new assets...
% \begin{enumerate}
%   \item If your work uses existing assets, did you cite the creators?
%     \answerTODO{}
%   \item Did you mention the license of the assets?
%     \answerTODO{}
%   \item Did you include any new assets either in the supplemental material or as a URL?
%     \answerTODO{}
%   \item Did you discuss whether and how consent was obtained from people whose data you're using/curating?
%     \answerTODO{}
%   \item Did you discuss whether the data you are using/curating contains personally identifiable information or offensive content?
%     \answerTODO{}
% \end{enumerate}
% 
% \item If you used crowdsourcing or conducted research with human subjects...
% \begin{enumerate}
%   \item Did you include the full text of instructions given to participants and screenshots, if applicable?
%     \answerTODO{}
%   \item Did you describe any potential participant risks, with links to Institutional Review Board (IRB) approvals, if applicable?
%     \answerTODO{}
%   \item Did you include the estimated hourly wage paid to participants and the total amount spent on participant compensation?
%     \answerTODO{}
% \end{enumerate}
% 
% \end{enumerate}

%
%
% * Maintain 100 characters / line. But keep sentence structure intact ---
%   Try not to break a line in the middle of a sentence.
%   Even if you do, break lines at a major clause boundary like ``and'', ``or''.
%
% * too much ``''s make the sentence look scattered and visually less recognizable. ``e.g.'' also.
%
% * \em, \bf, \it are all obsolete \TeX primitives, and it does not take effect properly ---
%   for example, {\bf {\it aaa}} shows ``aaa'' in italic but NOT IN BOLD.
%   Use \emph{}, \textit{}, \textbf{} and so on.
%
% * Always use \ff, \fd, \cea, or other abbreviation commands in common-abbrev.sty.
%   Find-and-replace is an error-prone operation that is inferior to custom commands
%   --- the amount of work needed for find-and-replace grows linearly with the number of usages,
%   and the probability of producing a consistent document shrinks exponentially.
%   Inconsistent notations would not be the major reason for rejection, but
%   they eventually harm the impression of the paper.
%
% * Use of footnotes should be minimized.
%
% * IPC2011 should always be \ipc . The definition can later be modified in abbrev.sty .
%
% * Prefer separated words over hyphened words, e.g., domain independent > domain-independent,
%   planner independent > planner-independent.
%
% * Table, Figure, Fig., should not be used directly.
%   Always use \refig and \reftbl etc, which are defined in common-general.sty .
%   When the development flag is enabled, direct use of \ref signals an error.
%
% * Caption must end with a period.
%
%

% How to write an intuitive paper:

% 1. There is a villain.
% 2. The villain has a weakness.
% 3. The hero defeats this villain by attacking the weakness.

% A good weakness tends to be theoretical and fundamental.
% Practical limitations can be a weakness, but does not provide motivation as strong as theoretical one.

\section{Available commands}

\paragraph{STRIPS}

\verb`\strips[options]` macro produces various formal definitions of classical planning.
Available options are:
\begin{itemize}
 \item \texttt{negative-preconditions},
 \item \texttt{disjunctive-preconditions},
 \item \texttt{conditional-effects},
 \item \texttt{action-costs},
 \item \texttt{derived-predicates},
 \item \texttt{optimal} (discuss optimality),
 \item \texttt{satisficing} (discuss satisficing search. Implies \texttt{optimal}),
 \item \texttt{agile} (discuss agile search. Implies \texttt{satisficing})
 \item \texttt{lifted} (discuss grounding),
 \item \texttt{unitcost} (clarifies all costs are 1. Implies \texttt{action-costs}).
\end{itemize}
Use it like \verb`\strips` \texttt{[derived-predicates,} \texttt{negative-preconditions]}.

\paragraph{Heuristics}

\verb`\heuristics[options]` macro describes heuristic functions and
introduces several heuristics specifies in the options.
The options can be put in two categories.

The first set of options add definitions for various classes of heuristics.
\begin{itemize}
 \item \texttt{perfect}: perfect heuristic $h^*$,
 \item \texttt{admissible} (implies \texttt{perfect}),
 \item \texttt{inadmissible} (implies \texttt{admissible}),
 \item \texttt{perfectsat}: perfect satisficing heuristics from \cite{kuroiwa2022biased} (implies \texttt{inadmissible}),
 \item \texttt{tdom}: $t$-dominance from \cite{kuroiwa2022biased} (implies \texttt{perfectsat}),
 \item \texttt{relaxation}: $h^+$ and relaxation heuristics.
\end{itemize}

The second set of options introduce known heuristics and their definitions.
Available heuristics are:
\begin{itemize}
 \item \texttt{ad} ($\ad$, additive heuristics)
 \item \texttt{hmax} ($\hmax$)
 \item \texttt{ff} ($\ff$)
 \item \texttt{gc} ($\gc$)
\end{itemize}

These options can take values between 0 and 2:
\begin{enumerate}
 \item If 0, the heuristics is omitted.
 \item If 1, it is added to the list of heuristics.
 \item If 2, its definitions are also added.
\end{enumerate}
Use them like \verb`\heuristics` \texttt{[ad=2,ff,perfect]}.


\def\stripsusage#1{
  \paragraph{\texttt{\textbackslash{}strips[#1]}}
  \strips[#1]
}

\def\heuristicusage#1{
  \paragraph{\texttt{\textbackslash{}heuristics[#1]}}
  \heuristics[#1]
}

\section{Examples}

\stripsusage{}
\stripsusage{lifted}
\stripsusage{derived-predicates}
\stripsusage{negative-preconditions}
\stripsusage{disjunctive-preconditions}
\stripsusage{conditional-effects}

\stripsusage{action-costs}
\stripsusage{unitcost}

\stripsusage{optimal}
\stripsusage{optimal,action-costs}
\stripsusage{satisficing}
\stripsusage{satisficing,action-costs}
\stripsusage{agile}
\stripsusage{agile,action-costs}


\heuristicusage{}

\heuristicusage{hmax,ad,ff,gc}

\heuristicusage{ad=2,ff}

\heuristicusage{perfectsat,relaxation}


\section{Minted}

\begin{minted}{cl}
 (print :hi)
\end{minted}

\section{Rule of Intuitive Writing}

\paragraph{Dichotomy}

Human minds prefer dichotomy like villain vs hero.
Concrete contrast with existing work is always necessary.

\paragraph{Consistency}

Our temporary memory is limited 7 chunks.
Papers exceeding this limit takes a lot longer to read and is painful to read.
Consistency greatly saves the memory usage.

One trick is to avoid a $N\times M$ structure and make it into a $N \times 1 + N \times (M-1)$ structure.
That is, initially, use a \emph{single} example to explain $N$ concepts (and put them into reader's memory)
then explain the rest of the $M-1$ examples, by which time the $N$ entries should have been in the permanent memory
thus does not consume the temporary memory.
An ADHD person has a temporary memory far fewer than 7 of a normal person.


\paragraph{Name by intent, not content}

This is a common advise also used in programming.

\paragraph{Cover several types of cognitions}

Human cognition has several types called VAK learning style.
VAK is for Visual, Auditory, Kinesthetic/tactile.
Make sure to appeal to all cognition types.

\begin{itemize}
 \item One is skewed toward visual cognition. An extreme example is a person with a photographic memory.
 \item Another is skewed toward sequential, language and auditory cognition.
 \item The last one is skewed toward Kinesthetic cognition (experience based).
\end{itemize}

\paragraph{Show, not Tell}

The basics of storytelling.

\paragraph{Tempo}

Many good movies have long introductions where nothing happens in the beginning, but all of them have sufficient foreshadowing.
People get bored and fails to engage if it lacks the elements that keep them interested.
Storytelling must get to the point with sufficient speed, otherwise it is difficult to keep the concentration.
Losing readers before getting to the core contribution would make all the effort meaningless, which anyone would avoid.

\paragraph{CRAP principle (Contrast, Repetition, Alignment, Proximity)}

One of the few principles of Graphic Design that is useful in presentation slides.

Make a color contrast.

Effective use of repetition of the same pattern
will lower the cognitive effort.

Align elements when they have orthogonal concepts.

Make similar, related things closer to each other,
and make different things farther.





\section{Introduction}

[The text is borrowed from a famous latex template for Kakenhi Grant
Proposal (Japanese version of NSF in US) which was slightly edited and translated into English.]

[Make sure to remove these texts before submitting a paper.
We know some accidents have already occurred in respectable universities during important grant submissions,
bringing confusion to the administrative staff, embarrassment to the researcher,
and some rumor among the department that the researcher lost sanity.]


The true purpose of this study is, in a nutshell, to fulfill the childhood dream of finding an elephant egg.

\section{Background}

The purpose of this research is
a multifaceted investigation of elephant egg shells from biological, chemical, theoretical, and engineering aspects.
Elephant egg shells weigh more than 80 kg.
It is necessary not only to support the weight of the elephant and its
nutrient source, the large mass of the yolk, but also the weight of the
parent elephant that warms the egg. For this reason, the elephant egg
shell is considered to have a structure that is entirely different
from that of bird's (Fig. 1) egg shell. Also, if the mechanism
of the elephant egg shell is clarified,

\begin{itemize}
 \item Elucidation of elephant ecology, understanding of dinosaur egg structure (biology),
 \item Elucidation of shell chemical formation reaction (chemistry),
 \item Research on the relationship between the atomic level structure of the shell and C60 and nanoclusters Research (physics),
 \item Artificially create an elephant shell and apply it to the body of a car (engineering)
\end{itemize}

The impact on science and society is immeasurable.

\section{Method}

We traveled around the world to find elephant eggs. This has also been a dream since I was a kid.

\section{Empirical Evaluation}

Elephant eggs are phantastic.
Elephant eggs are phantastic.
Elephant eggs are phantastic.
Elephant eggs are phantastic.
Elephant eggs are phantastic.
Elephant eggs are phantastic.
Elephant eggs are phantastic.
Elephant eggs are phantastic.

\section{Related Work}

According to \citet{folbre1997future},
In ``Horton Hatches the Egg'' \cite{seuss1968horton},
a tired mother bird asked Horton the elephant to give her a break and
sit on the nest instead, which resulted in a chick with a small trunk and
elephant ears.
\citet{kipling1983elephant} discuss how the elephant got its long trunk.
\citet{cooper2001complete} analyzed the mitochondria DNA sequence of the extinct Elephant-Bird species,
such as \emph{Aepyornis maximus}.
\citet{carlqvist2003theory} studied the astronomical object called Twisted Elephant Trunks
based on the magnetic flux ropes between molecular clouds.

Existing work studied the eggs of the blue whale.
While blue whales and elephants are mammals, they live in completely different habitats.

\section{Conclusion}

We could not find Elephant's eggs.

\section*{Checklist}

Elephant egg ES cells are not cultured or elephant clones are not
generated. Since elephant individuals are not taken out of the field,
they do not conflict with the Washington Convention and the Convention
on Biological Diversity. In addition, since no recombination experiment
is conducted, it does not conflict with the Cartagena Protocol.

%%% BEGIN INSTRUCTIONS %%%
% The checklist follows the references.  Please
% read the checklist guidelines carefully for information on how to answer these
% questions.  For each question, change the default \answerTODO{} to \answerYes{},
% \answerNo{}, or \answerNA{}.  You are strongly encouraged to include a {\bf
% justification to your answer}, either by referencing the appropriate section of
% your paper or providing a brief inline description.  For example:
% \begin{itemize}
%   \item Did you include the license to the code and datasets? \answerYes{See Section~\ref{gen_inst}.}
%   \item Did you include the license to the code and datasets? \answerNo{The code and the data are proprietary.}
%   \item Did you include the license to the code and datasets? \answerNA{}
% \end{itemize}
% Please do not modify the questions and only use the provided macros for your
% answers.  Note that the Checklist section does not count towards the page
% limit.  In your paper, please delete this instructions block and only keep the
% Checklist section heading above along with the questions/answers below.
%%% END INSTRUCTIONS %%%

% \begin{enumerate}
%
% \item For all authors...
% \begin{enumerate}
%   \item Do the main claims made in the abstract and introduction accurately reflect the paper's contributions and scope?
%     \answerTODO{}
%   \item Did you describe the limitations of your work?
%     \answerTODO{}
%   \item Did you discuss any potential negative societal impacts of your work?
%     \answerTODO{}
%   \item Have you read the ethics review guidelines and ensured that your paper conforms to them?
%     \answerTODO{}
% \end{enumerate}
%
% \item If you are including theoretical results...
% \begin{enumerate}
%   \item Did you state the full set of assumptions of all theoretical results?
%     \answerTODO{}
% 	\item Did you include complete proofs of all theoretical results?
%     \answerTODO{}
% \end{enumerate}
%
% \item If you ran experiments...
% \begin{enumerate}
%   \item Did you include the code, data, and instructions needed to reproduce the main experimental results (either in the supplemental material or as a URL)?
%     \answerTODO{}
%   \item Did you specify all the training details (e.g., data splits, hyperparameters, how they were chosen)?
%     \answerTODO{}
% 	\item Did you report error bars (e.g., with respect to the random seed after running experiments multiple times)?
%     \answerTODO{}
% 	\item Did you include the total amount of compute and the type of resources used (e.g., type of GPUs, internal cluster, or cloud provider)?
%     \answerTODO{}
% \end{enumerate}
%
% \item If you are using existing assets (e.g., code, data, models) or curating/releasing new assets...
% \begin{enumerate}
%   \item If your work uses existing assets, did you cite the creators?
%     \answerTODO{}
%   \item Did you mention the license of the assets?
%     \answerTODO{}
%   \item Did you include any new assets either in the supplemental material or as a URL?
%     \answerTODO{}
%   \item Did you discuss whether and how consent was obtained from people whose data you're using/curating?
%     \answerTODO{}
%   \item Did you discuss whether the data you are using/curating contains personally identifiable information or offensive content?
%     \answerTODO{}
% \end{enumerate}
%
% \item If you used crowdsourcing or conducted research with human subjects...
% \begin{enumerate}
%   \item Did you include the full text of instructions given to participants and screenshots, if applicable?
%     \answerTODO{}
%   \item Did you describe any potential participant risks, with links to Institutional Review Board (IRB) approvals, if applicable?
%     \answerTODO{}
%   \item Did you include the estimated hourly wage paid to participants and the total amount spent on participant compensation?
%     \answerTODO{}
% \end{enumerate}
%
% \end{enumerate}
